\documentclass[]{article}

%%%%%%%%%%%%%%%%%%%
% Packages/Macros %
%%%%%%%%%%%%%%%%%%%
\usepackage{amssymb,latexsym,amsmath, mathtools}     % Standard packages
\usepackage[brazil]{babel}
\usepackage[utf8]{inputenc}

%%%%%%%%%%%
% Margins %
%%%%%%%%%%%
\addtolength{\textwidth}{1.0in}
\addtolength{\textheight}{1.00in}
\addtolength{\evensidemargin}{-0.75in}
\addtolength{\oddsidemargin}{-0.75in}
\addtolength{\topmargin}{-.50in}
\linespread{1.3}

%%%%%%%%%%
%%macros%%
%%%%%%%%%%
\newcommand{\defi}{\textbf} % \textit

%%%%%%%%%%%%%%%%%%%%%%%%%%%%%%
% Theorem/Proof Environments %
%%%%%%%%%%%%%%%%%%%%%%%%%%%%%%
\newtheorem{theorem}{Teorema}
\newtheorem{conjecture}[theorem]{Conjectura}
\newenvironment{proof}{\noindent{\bf Prova:}}{$\hfill \Box$ \vspace{10pt}}  
\newenvironment{definition}{\noindent{\bf Definição:}}


%%%%%%%%%%%%
% Document %
%%%%%%%%%%%%
\begin{document}

\title{Resumo de teoremas de kernels em digrafos}
\author{Arthur Rodrigues}
\date{}
\maketitle

\section{Definições}

Um digrafo é chamado de \defi{kernel perfeito} se todo subgrafo induzido possui kernel.

Dado um circuito $C$, uma \defi{corda} de $C$ é uma aresta cujo vértices adjacentes estão em $C$.
O \defi{comprimento} da corda é a distância dos seus vértices adjacentes \textit{no circuito}.
Uma corda é \defi{curta} se seu comprimento é igual a $2$. Duas cordas $c=(a,b)$ e $z=(x,y)$ \defi{cruzam}, se seus vértices aparecem na ordem
$a$, $x$, $b$ e $y$ no circuito.


Um digrafo $D$ é chamado de \defi{super-orientação} de um grafo $G$, se $D$ é obtido de $G$ substituindo suas arestas por arcos, podendo ser duplos.

Um grafo $G$ é \defi{kernel-solucionavel} (ou \defi{quase-perfeito}) se, para toda super-orientação de $D$ de $G$ em que todo clique de $D$ possui um kernel, $D$ também possui um kernel.

Seja $G$ um grafo, 
$B=  (v_0,\dots,v_n)$ uma sequência de vértices de $G$ e $C= (C_0,\dots,C_n)$ uma sequência de cliques.
Dizemos que $G'$ é uma substituição de $B$ por $C$ em $G$ se $G'$ é obtido a partir de $(G - B) \cup C$ adicionando 
as arestas de $C_i \in C$ e $\{uc, c \in V(C_i) : \exists v \in B : uv \in E(G) \}$.
Chamamos $G'$ de um \defi{blow-up} de $G$.

Seja $D$ um digrafo e seja $S \subset A(D)$. Dizemos que um conjunto de arcos $T$ é uma \defi{orientação} (ou \defi{reorientação}) de $S$ se, para todo $f\in T$,$f^-$ ou $f\in S$.
Note que é possível $f$ e $f^-$ existir em $T$, mas só um deles existir em $S$. Uma orientação é \defi{parcial} se $f \in T$, então  $f^- \notin T$ e ou $f \in S$ ou 
$f^- \in S$. Uma orientação~$T$ é \defi{total} se $f \in S$, então $f \in T$ ou $f^- \in T$.

\section{Teoremas}
\begin{theorem}[von Neumann]
Todo digrafo acíclico é kernel perfeito. Além disso,seu kernel é único.
\end{theorem}

\begin{proof}
Ordem topológica ou por indução em número de vértices
\end{proof}


\begin{theorem}[Richardson]
Todo digrafo sem circuito ímpar é kernel perfeito.
\end{theorem}
\begin{proof}
Se é fortemente conexo, usa busca em largura.

Se não:
Seja $D$ um digrafo sem circuitos ímpares com $n$ vértices. 
Podemos supor que $D$ não seja fortemente conexo. 
Seja $D'$ um componente fortemente conexo de $D$ do 
qual não existem arcos saindo dele. 
Pela hipótese de indução,
$D'$ possui um kernel $K'$. 
Seja $D''= D-N^-[K]$. Pela hipótese de indução, $D''$ possui um kernel $K''$.
Como não existem $(K',K'')$-arcos nem $(K'',K')$-arcos,$K' \cup K''$ é um kernel de
$D$, pois todo vértice de $D$ pertence a $K'$,$K''$ ou é absorvido por eles e, além disso, $K'$ e $K''$ são independentes.
\end{proof}

“O teorema direciona a atenção do estudo da existênciade kernel para digrafos que contém circuitos ímpares.”


\begin{theorem}[Teorema forte de grafos perfeitos]
Um grafo é perfeito se ele não possui circuitos impares com ao menos cinco vértices nem complementos desses como subgrafos induzidos.
\end{theorem}

\begin{theorem}
Grafos perfeitos são kernel-solucionáveis.
\end{theorem}

A recíproca é mais difícil.

\begin{theorem}
Grafos de linha kernel-solucionáveis são perfeitos.
\end{theorem}

\begin{theorem}
Se todo blow-up $G'$ de $G$ é kernel-solucionavel, então G é perfeito.
\end{theorem}

“Pode-se mostrar que ambas as conjecturas seguem do Teorema Forte dos Grafos Perfeitos. Por conta de ser uma prova muito extensa, a omitimos.”

\begin{conjecture}[Meyniel]
Se todo circuito ímpar de um digrafo $D$ possui duas cordas, então $D$ é kernel-perfeito.
\end{conjecture}

A conjectura se mostrou falsa.


\begin{theorem}
Se todo circuito de um digrafo $D$ possui um arco simétrico, o digrafo é kernel perfeito.
\end{theorem}



\end{document}
